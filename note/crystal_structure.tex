\chapter{Structure of Crystal}

\section{晶体的 X 射线衍射}
    \subsection{引言}
        \begin{enumerate}
            \item 选用 X 射线的原因: 波长为 $10^{-8}\ \mathrm{cm}$, 小于晶格尺度
            \item 想法的提出: 劳厄, 1917
            \item 布拉格父子: 布拉格定律, 1912
        \end{enumerate}

    \subsection{衍射的基本方法}

    \subsection{衍射方程}
        \paragraph{布拉格反射公式 \\}

        \paragraph{劳厄衍射方程 \\}

        \paragraph{由劳厄衍射方程导出布拉格反射公式 \\}

        \paragraph{反射球 \\}
            \hspace*{2em} 在能量守恒假设下, 入射波矢和反射波矢的模方始终是相等的, 可知满足条件的 $\bm{k}$ 一定在 $\bm{k}_0$ 划出的一个球面上, 这个球面被称为\,\textbf{反射球}\,. 由劳厄衍射方程:
                \begin{equation}
                    \bm{k} - \bm{k}_0 = n \bm{G}
                \end{equation}
            可知, \CJKunderwave{满足衍射条件条件的倒格矢\ $n \bm{G}$ 一定是反射球的一条弦.}
