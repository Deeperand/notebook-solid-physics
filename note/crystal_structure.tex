\chapter{Structure of Crystal}
\section{一些晶格的实例}
    \subsection{引言}
        \paragraph{固体的结构 \\}
            \hspace*{2em} 固体总体上可以分为\,\textbf{晶体}, \textbf{非晶体}, \textbf{准晶体}. 晶体的特点是具有长程有序性; 非晶体则不具有长程有序性; 准晶体具有结构性 (如一些拼接图形), 但同样不具有长程有序性.

        \paragraph{固体物理学的发展历史 \\}
            \hspace*{2em} 较为重要的历史事件有:
                \begin{enumerate}
                    \item 18 世纪: 阿羽依, 规则几何外形 $\Leftrightarrow$ 内部规则性
                    \item 1850 年: 布拉伐 (Bravais) 提出空间点阵学说, 提供了经验规律
                    \item 1912: 劳厄: 发现 X 射线通过晶体的衍射现象, 证实了晶体内部原子周期性排列的结构
                    \item 1913: 布拉格 (Bragg) 父子建立了晶体结构分析的基础
                    \item 二次大战后的 *中子衍射技术*
                \end{enumerate}

    \subsection{晶格的分类}
        \paragraph{晶格的概念 \\}
            \hspace*{2em} 首先先引入晶格的一般性概念:
                \begin{Concept}[晶格]
                    晶体中原子的排列的具体形式一般称为晶体格子, 或简称为\,\textbf{晶格}.
                \end{Concept}

        \paragraph{简单立方堆积 (SC) \\}

        \paragraph{体心立方堆积 (BCC) \\}

        \paragraph{面心立方堆积 (FCC) \\}

        \paragraph{六角密排 (HCP) \\}

        \paragraph{金刚石型排列 \\}

        \paragraph{小结 \\}
            \hspace*{2em} 表格\ref{tab: 晶格的分类}总结了本节所提到的晶格及实例 (布拉伐格子的概念见后面):

                \begin{table}[H]
                    \caption{晶格的分类}\label{tab: 晶格的分类}

                    \centering
                    \begin{tabular}{|c|c|c|c|c|c|}
                        \hline
                        \textbf{元素晶体}   & SC   & BCC & FCC  & HCP    & 金刚石 \\ \hline
                        \textbf{简单化合物} & CsCl &     & NaCl & 纤锌矿 & 闪锌矿 \\ \hline
                        \textbf{布拉伐格子} & SC   & BCC & FCC  & HCP    & FCC \\ \hline
                    \end{tabular}
                \end{table}

\section{晶体的 X 射线衍射}
    \subsection{引言}
        \begin{enumerate}
            \item 选用 X 射线的原因: 波长为$10^{-8}$ cm, 小于晶格尺度
            \item 想法的提出: 劳厄, 1917
            \item 布拉格父子: 布拉格定律, 1912
        \end{enumerate}

    \subsection{衍射的基本方法}

    \subsection{衍射方程}
        \paragraph{布拉格反射公式 \\}

        \paragraph{劳厄衍射方程 \\}

        \paragraph{由劳厄衍射方程导出布拉格反射公式 \\}

        \paragraph{反射球 \\}
            \hspace*{2em} 在能量守恒假设下, 入射波矢和反射波矢的模方始终是相等的, 可知满足条件的 $\bm{k}$ 一定在 $\bm{k}_0$ 划出的一个球面上, 这个球面被称为\,\textbf{反射球}\,. 由劳厄衍射方程:
                \begin{equation}
                    \bm{k} - \bm{k}_0 = n \bm{G}
                \end{equation}
            可知, \CJKunderwave{满足衍射条件条件的倒格矢\ $n \bm{G}$ 一定是反射球的一条弦.}
