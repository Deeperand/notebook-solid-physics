% input this file: \input{/Users/he/Documents/LaTeX/Notebooks/notebook_customize.tex}
% documentclass need: ctex related (ctexart, ctexbook, ...)
% test

% global setting
    \ctexset{autoindent = 0em} % cancel autoindent

    % keep appropriate distance while there being a large equation in a row
    \setlength\lineskiplimit{0.3em} 
    \setlength\lineskip{0.3em}

    % change the symbol size in math mode
    \DeclareMathSizes{10.54}{10.54}{5.27}{3.16}



% other packages
%    \usepackage{pdfsync}
    \usepackage{import}
    \usepackage{xifthen}
    \usepackage{pdfpages}
    \usepackage{transparent}

    % package in drawing
    \usepackage{tikz}
    \usetikzlibrary{scopes} % let use of scopes environment easier
    \usetikzlibrary{optics} % use library: optics (made the draw of optic much easier)
    \usetikzlibrary{arrows.meta} % more type of arrow
    \usetikzlibrary{intersections} % calcuate intersection
    \usetikzlibrary{calc}
    \usetikzlibrary{trees} % tree graph
    \usetikzlibrary{chains} % chain plot
    \usetikzlibrary{matrix}

    % package using in text mode
    \usepackage[a4paper, left=2.54cm, right=2.54cm, top=2.54cm, bottom=2.54cm]{geometry}
    \usepackage{listings}
    \usepackage{xcolor}
    \usepackage{enumitem}
    \usepackage{graphicx}
    \usepackage{float}
    \usepackage{booktabs}
    \usepackage{caption} % control the caption format of picture
    \usepackage{longtable} % auto-change-page-table
    \usepackage{nameref} % provide command: \nameref
    \usepackage{fontspec} % font select and set
    \usepackage{CJKfntef} % enhanced under modify, the parameter can cancel replacing '\emph' with '\CJKunderline'
    \usepackage[normalem]{ulem} % 除可以用于下划线修饰以外, 还提供了指令 \bgroup 与 \markoverwith \Ulon, 可用于自定义文字上的风格.
    \usepackage{cprotect} % make the use of `verb` more convenient
    \usepackage{multirow} % allow generate cross-row table (command "\multirow")
    \usepackage{changepage} % indent the entire paragraph
    \usepackage{theorem} % control theorem enviroment

    % package using in math mode
    \usepackage{amsmath}
        \allowdisplaybreaks[4] % allowed multiline equation can change page
        % \everymath{\displaystyle} % set default math style as displaystyle
    \usepackage{amssymb}
    \usepackage{mathtools}
    \usepackage{mathrsfs} % provide the font of \mathscr
    \usepackage{tensor}
    \usepackage{upgreek}
    \usepackage{extarrows}
    \usepackage{yhmath}
    \usepackage{nccmath} % realign the equation
    \usepackage{xfrac} % split level fractions
    \usepackage{bbm} % hollow number
    \usepackage{bm} % bold, but keep italic

% self define
    \definecolor{CodeGrey}{RGB}{240, 240, 240} % the grey used in code block or single code
    \definecolor{InLineCodeColor}{RGB}{94, 160, 40} % the grey used in code block or single code

% command/environment building and renewing
    % text
        % env
            % Indent
                \newenvironment{Indent}
                {\begin{adjustwidth}{1em}{0em}}%
                {\end{adjustwidth}}

            % Notation
                \newcounter{Notation}[chapter]
                \newcommand\ChapterNum{\arabic{chapter}}
                \newcommand\NotationNum{\arabic{Notation}}

                \newenvironment{Notation}
                {\stepcounter{Notation}%
                \begin{Indent} \textbf{Notation \ChapterNum.\NotationNum\ :}%
                    \begin{enumerate}}%
                    {\end{enumerate}%
                \end{Indent}}

            % Solve: used to "proof", "solve" and so on. The command \Box is better than $\square$
                \newcommand\QedSymbol{\ensuremath{\Box}}

                \newenvironment{Solve}
                [1][解:]%
                {\upshape\textbf{#1 \\}}%
                {\begin{flushright}\QedSymbol\end{flushright}}

            % Example (If use "\slshape" or "\itshape", the Chiniese will use "\kaishu". However, use "\itshape" will let it hard to distinct main body and math equaion if language is English. So use "\slshape" is better.
                \newcommand\ExampleSymbol{\ensuremath{\blacksquare}}

                \newcounter{Example}[chapter]
                \newcommand\ExampleNum{\arabic{Example}}

                \newenvironment{Example}
                [1][]%
                {\stepcounter{Example}%
                    \begin{Indent}%
                        \textbf{Example \ChapterNum.\ExampleNum\ #1}%
                            \begin{Indent}%
                            \slshape}%
                            {\end{Indent}%
                            \begin{flushright}\ExampleSymbol\end{flushright}%
                    \end{Indent}}%



        % command
            \newcommand{\Code}[1]{\lstinline[basicstyle=\color{InLineCodeColor}\ttfamily]{#1}} % in-line code
            \newcommand{\CodeA}[1]{\textcolor{InLineCodeColor}{\texttt{#1}}} % in-line code, box version, can be used at table
            \newcommand{\CodeB}[1]{\colorbox{CodeGrey}{\rule{0pt}{1ex}\texttt{#1}}} % code with box, had the less height "ex", which means even the punctuation like comma had the height "ex".
            \newcommand{\completesolving}{\begin{flushright}$\square$\end{flushright}} % the square
            \newcommand{\CompleteExample}{\begin{flushright}$\blacksquare$\end{flushright}} % the black square
            \newcommand\DoubleQuote[1]{``#1''} % double quote
            \newcommand\SingleQuote[1]{`#1'} % single quote

        % parenthesis (为了避免因只输入单侧括号而造成文本编辑器错误的语法高亮, 不过这种情况较为少见)
            \newcommand\BraPL{(} % bracket: left parenthesis
            \newcommand\BraPR{)} % bracket: right parenthesis
            \newcommand\BraSL{[} % bracket: left square parenthesis
            \newcommand\BraSR{]} % bracket: right square parenthesis
            \newcommand\BraBL{\{} % bracket: left brace
            \newcommand\BraBR{\}} % bracket: right brace

        % font
            \newfontfamily\Zapfino{Zapfino}

    % math
        % command
            \newcommand{\FDEq}[1]{\fbox{$\displaystyle #1$}} % framed display-style equation
        % notation
            \newcommand{\Dbar}{\mathrm{d} \hspace*{-0.15em}\bar{}\hspace*{0.1em}} % \mathrm{d} with bar
            \newcommand\MaE{\mspace{2mu}\mathrm{e}\mspace{2mu}} % "Ma" is the abbreviation of "Math", "E" means natural constant "e"
            \newcommand\MaPI{\mspace{2mu}\uppi\mspace{2mu}} % ratio of the circumference of a circle to its diameter

    %set the display style of enumerate number
    \renewcommand\theenumi{\arabic{enumi}}
    \renewcommand\labelenumi{\theenumi).}
    \renewcommand\theenumii{\arabic{enumii}}
    \renewcommand\theenumiii{\Alph{enumiii}}
    \renewcommand\theenumiv{\Roman{enumiv}}

    % theorem environment
        % length before and after theorem environment
        \setlength{\theorempreskipamount}{3em}
        \setlength{\theorempostskipamount}{3em}

        % command define
            \makeatletter
            \@ifclassloaded{ctexbook}{%
                \newtheorem{ClaimOriginal}{Claim}[chapter]%
                \newtheorem{ConceptOriginal}{Concept}[chapter]%
                \newtheorem{DefinitionOriginal}{Definition}[chapter]%
                \newtheorem{FormulaOriginal}{Formula}[chapter]%
                \newtheorem{LawOriginal}{Law}[chapter]%
                \newtheorem{PropertyOriginal}{Property}[chapter]%
                \newtheorem{TheoremOriginal}{Theorem}[chapter]%

                % indent theorem enviroment
                \newenvironment{Claim}%
                [1][]%
                {\vspace{0.6em}\begin{Indent}%
                    \begin{ClaimOriginal}[#1]}%
                    {\end{ClaimOriginal}%
                \end{Indent}\vspace{0.6em}}

                \newenvironment{Concept}%
                [1][]%
                {\vspace{0.6em}\begin{Indent}%
                    \begin{ConceptOriginal}[#1]}%
                    {\end{ConceptOriginal}%
                \end{Indent}\vspace{0.6em}}

                \newenvironment{Definition}%
                [1][]%
                {\vspace{0.6em}\begin{Indent}%
                    \begin{DefinitionOriginal}[#1]}%
                    {\end{DefinitionOriginal}%
                \end{Indent}\vspace{0.6em}}

                \newenvironment{Formula}%
                [1][]%
                {\vspace{0.6em}\begin{Indent}%
                    \begin{FormulaOriginal}[#1]}%
                    {\end{FormulaOriginal}%
                \end{Indent}\vspace{0.6em}}

                \newenvironment{Law}%
                [1][]%
                {\vspace{0.6em}\begin{Indent}%
                    \begin{LawOriginal}[#1]}%
                    {\end{LawOriginal}%
                \end{Indent}\vspace{0.6em}}

                \newenvironment{Property}%
                [1][]%
                {\vspace{0.6em}\begin{Indent}%
                    \begin{PropertyOriginal}[#1]}%
                    {\end{PropertyOriginal}%
                \end{Indent}\vspace{0.6em}}

                \newenvironment{Theorem}%
                [1][]%
                {\vspace{0.6em}\begin{Indent}%
                    \begin{TheoremOriginal}[#1]}%
                    {\end{TheoremOriginal}%
                \end{Indent}\vspace{0.6em}}
            }{}
            \makeatother

% initialize the setting
    % avoid too large interval between picture's caption and picture
    \captionsetup{skip = 5pt}

    \setcounter{secnumdepth}{5}
    \ctexset{
        section/name = \S\,,
        subsection/beforeskip = 1.3ex plus 0.4ex minus 0.08ex, % 40% of original
        subsection/afterskip = 0.375ex plus 0.05ex, % 25% of original
        paragraph/aftertitle = \hspace{-0.125em},
        paragraph/aftername = \hspace{0.3em},
        paragraph/number = \arabic{paragraph}.,
        paragraph/beforeskip = 0.4875ex plus 0.15ex minus 0.03ex, % 15% of original
        paragraph/afterskip = 0.15em, % 15% of original
        paragraph/hang = false
    }

    \lstset{
    tabsize=4, % size of tab
    xleftmargin=2em, % distance between frame and paper edge
    xrightmargin=2em,
    framexleftmargin=1.5em, % are the dimensions which are used additionally to framesep to make up the margin of a     frame (distance between code and frame)
    framexrightmargin=1.5em, 
    basicstyle=\ttfamily,
    breaklines=true,
    columns=flexible,
    numbers=left,                                        % 在左侧显示行号
    numberstyle=\color{gray},                       % 设定行号格式
    numbersep=5pt,
    frame=none,                                          % 不显示背景边框
    backgroundcolor=\color{CodeGrey},               % 设定背景颜色
    keywordstyle=\color[RGB]{40,40,255},                 % 设定关键字颜色
    numberstyle=\footnotesize\color{darkgray},
    commentstyle=\ttfamily\color[RGB]{49,150,49},                % 设置代码注释的格式
    stringstyle=\rmfamily\slshape\color[RGB]{128,0,0},   % 设置字符串格式
    showstringspaces=false                               % 不显示字符串中的空格
    }

    \setlist[enumerate]{
        itemsep = 0pt,
        topsep = 0pt,
        partopsep = 0pt,
        parsep = 0pt,
        labelsep = 0.3em,
    }
